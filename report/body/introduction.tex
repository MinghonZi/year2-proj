\chapter{Introduction}

\section{Background}

There are currently three types of mobile robots: wheeled robots, tracked robots, and footed robots. Compared with the other two robots, the footed robot has outstanding advantages. For example, on natural terrain, a footed robot can easily move on sand, hard or soft ground with similar efficiency ~\cite{ref:1C2D}. It can still move even when it encounters discontinuous terrain. Wheeled robots can move efficiently only on flat and hard surfaces. Although tracked robots have certain off-road performance, their efficiency in harsh terrain conditions is still not as good as that of footed robots. In addition, In addition, in terms of slippage and jamming, since the actions of stepping on and raising legs of the footed robot are perpendicular to the ground and will not interfere with the ground, the footed robot can easily walk on soft and muddy roads ~\cite{ref:1C2D}. In contrast, the turning of the wheels of a wheeled robot causes the body to sink and become difficult to move. Therefore, the research on footed robots is a hot topic in the current scientific field.

The motion of the legs of a footed robot is quite complex, and in this project we refer to the method proposed by Denavit and Hartenberg in 1955. This method uses triangular relations and kinematics algorithms to derives the kinematic model ~\cite{ref:1C2D}.

\section{Objectives}

The objective of this project is to complete the simulation of the posture adjustment motions of the quadruped robot on the Pybullet platform. The motions simulated in the three directions are pitch, roll, and yaw motion. The project will develop programs for three actions respectively. Finally, three programs should be combined to complete the posture adjustment of the quadruped robot on three axes in space. In addition, the height adjustment of the robot can be regarded as a part of the posture adjustment. The height adjustment program is combined into the posture adjustment program to achieve the additional goal of adjusting the posture of the quadruped robot at any height.
