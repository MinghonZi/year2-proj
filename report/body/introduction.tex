\chapter{Introduction}

\section{Rationale}

There are currently three types of mobile robots: wheeled robots, crawler robots, and footed robots. It is difficult to see the difference in the moving efficiency of the three mobile robots on an ideal level of the ground. However, in reality, there are more bumpy and slippery scenarios in which the mobility efficiency of the footed robot is unmatched by the other two robots. Therefore, the application prospect of the footed robot is very broad.

The movement of a footed robot in a complex environment is inseparable from the ability to adjust the posture every moment. This ability helps the robot to maintain balance and complete movement during motion. This project will use the forward and inverse kinematics algorithm to solve the robot joint coordinates before and after posture adjustment, and control the model in the Pybullet simulation environment to complete the simulation of the posture adjustment of the quadruped robot.

\section{Objectives}

The objective of this project is to develop a posture adjustment program for a quadruped robot in three directions in three-dimensional space. The motions completed by the posture adjustment in the three directions are pitch motion, roll motion, and yaw motion. Finally, three programs should be combined to complete the posture adjustment of the quadruped robot in any direction in space.

In addition, the height adjustment of the robot can also be regarded as a part of the posture adjustment. The height adjustment program is combined into the posture adjustment program to achieve the additional goal of adjusting the posture of the quadruped robot at any height.
